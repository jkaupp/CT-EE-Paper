It draws from the substantial literature, drawing broadly from international agreements of professional engineering profiles\cite{internationalengineeringalliance}, initiatives in engineering education\cite{crawley2011cdio}, ways of thinking in engineering (design and systems thinking) \cite{Dorst:2010tc, Dym:2005gy, Frank:2001ut}and ways of reasoning in engineering \cite{Stein:2011hr, Paul:2006kv}. 

Underlying the majority of models of engineering thinking is the application of thinking and reasoning to solve complex engineering problems. These types of problems are representative of design problems routinely encountered in professional practise, and can be characterized by the following\cite{internationalengineeringalliance}: 

\begin{itemize}
\item Cannot be resolved without in-depth engineering knowledge that allows a fundamentals-based, first principles analytical approach.
\item Involve wide-ranging or conflicting technical, engineering and other issues.
\item Have no obvious solution and require abstract thinking, originality in analysis to formulate suitable models
\item Involve infrequently encountered issues
\item Are outside problems encompassed by standards and codes of practice for professional engineering
\item Involve diverse groups of stakeholders with widely varying needs
\item Are high level problems including many component parts or sub-problems
\item Have significant consequences in a range of contexts
\item Require judgement in decision making
\end{itemize}

Consistent with the above, our framework views thinking and reasoning in engineering is conducted within a problem solving context and should embody the three distinguishing characteristics of engineering learning: the use of representations, alignment with professional practices, and emphasis on design\cite{Johri:2014tk}. This context, along with alignment with disciplinary perspectives on learning should naturally lead to the assessment of critical thinking in engineering in an authentic fashion; which emphasizes design principles to develop solutions to  complex engineering problems, which are well aligned and representative of professional practise.
