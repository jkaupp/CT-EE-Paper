\subsection{Critical thinking assessments}

Six assessment approaches, which include standardized instruments (Collegiate Learning Assessment Plus, Critical Thinking Assessment Test, Cornell Critical Thinking Test and the International Critical Thinking Test), Meta-rubrics (VALUE Rubrics), and an approach utilizing model eliciting activities.  These approaches were selected due to their use in higher education/cite{Stein:2011hr,Stein:2007wd,Benjamin:2009uo,Benjamin:2009ud,rhodes_emerging_2012,Rhodes:2013ww,Clark:2012ur}, engineering education/cite{Ralston:2010vh,Ralston:2011kh,Ralston:2014vc,Ennis:1985vj}, and the authors previous work/cite{Kaupp:2014un,Kaupp:2013va}.



Each assessment is based on a conception of critical thinking, be it an explicit framework, working definition or both.  Due to this, there are considerable differences between each assessment ranging from the type of assessment (paper, online), mode of the assessment (open-ended,multiple choice, mixed), constructs and discrete skills measured, scoring procedures, cost and psychometric properties.  Each of these aspects will be addressed alongside considerations regarding the  authenticity, practicality, sustainability of each assessment and it's alignment with the previously presented definition of critical thinking in engineering, illustrated below in figure 3.
