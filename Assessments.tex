\subsection{Critical thinking assessments}

In this paper six assessment approaches, which include standardized instruments (Collegiate Learning Assessment Plus, Critical Thinking Assessment Test, Cornell Critical Thinking Test and the International Critical Thinking Test), Meta-rubrics (VALUE Rubrics), and an approach utilizing model eliciting activities.  These approaches were selected due to their use in higher education/cite{Stein:2011hr,Stein:2007wd,Benjamin:2009uo,Benjamin:2009ud,Rhodes:2012wt,Rhodes:2013ww,Clark:2012ur}, engineering education/cite{Ralston:2010vh,Ralston:2011kh,Ralston:2014vc,Ennis:1985vj}, and the authors previous work/cite{Kaupp:2014un,Kaupp:2013va}.

Each assessment is based on a conception of critical thinking, be it an explicit framework, working definition or both.  Due to this there are considerable differences between each assessment ranging from the type of assessment (paper, online), mode of the assessment (open-ended,multiple choice, mixed), constructs and discrete skills measured, scoring procedures, cost and psychometrics.   Each approach is replete with it’s own strengths and weaknesses, and in the following sections each 
