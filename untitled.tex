Engineers solve problems.  From simple to complex, constrained to open-ended, everyday to specific, solving problems is one of the fundamental abilities of the discipline.  Underlying problem solving is the reflective, self-regulated judgement, analysis and reasoning exemplified by critical thinking (CT).  Critical thinking is considered by many as an essential outcome of a higher education and is highly valued by employers, educators and policymakers.  Yet, many  programs claim to develop critical thinking skills, but this development is informal, non-continuous, holistic in nature and infrequently assessed\cite{Paul:1997ti, Ahern:2012ey}.  This is usually attributed to the lack of consensus and debates amongst CT scholars and researchers on a number of issues.  These range from a lack of a consensus operational definition, disagreement on domain-specificity\cite{Ennis:1989bm, McPeck:1990is}, effective instructional techniques\cite{Tsui:2002cc} and the difficulty of assessing a latent, complex cognitive and metacognitive skill\ cite{Ennis:1993us}
{Ennis:1993us, Halpern:2003tt}.   

