\subsection{Model Eliciting Activities}

MEAs have been used in engineering education at the university level for the past decade {DiefesDux:2004fq, Moore:2004dc, Shuman:2008tq, Frank:2012ut, Kaupp:2013va}. MEAs have shown promising results in developing students’ topical conceptual understanding, information fluency, problem solving and communication skills{Shuman:2008tq}. MEAs require students to draw upon prior knowledge and often help to identify and address misconceptions in the course of learning and promote connections between information.

There is no explicit framework of thinking skills embedded with the MEAs, leaving the instructor free to carefully adapt and align a selected framework by which to provide scaffolding, structure and guidance for students to develop and apply to the process by which they solve the MEA. The MEAs are designed according to a set of six principles outlined below{Anonymous:Ekrx99y7, Moore:2004dc}: 

1)	Model construction: The activity requires the construction of an explicit description, explanation or procedure for a mathematically significant situation.
2)	Reality: Requires the activity to be posed in a realistic engineering context and to be designed so that the students can interpret the activity meaningfully from their different levels of mathematical ability and general knowledge.
3)	Self-assessment: The activity contains criteria that students can identify and use to test and revise their current ways of thinking.
4)	Model documentation: Students are required to create some form of documentation that will reveal explicitly how they are thinking about the problem situation.
5)	Construct share-ability and re-usability: Requires students to produce solutions that are shareable with others and modifiable for other engineering situations.
6)	Effective prototype: Ensures that the model produced will be as simple as possible yet still mathematically significant for engineering purposes.
 
MEA instruction places a considerable emphasis on the process used to solve the problem and the reasoning and thinking students used to develop their solutions rather than on the product of that methodology. The solution of an MEA requires participants to apply and combine multiple engineering, physics or mathematical concepts drawn from their educational experience and previous background to formulate a general mathematical model that can be used to solve the problem. Students typically employ an iterative process approach to the MEA, first generating a model, testing the model and revising the model to develop a suitable solution{Lesh:2003ut}. The students’ solutions to the MEA typically take the form of a comprehensive report outlining the process used to generate their solution to the problem. 

There have been several studies investigating the impact of MEA instruction on student learning outcomes and general skill development. These studies have shown that MEAs:

1)	Encourage a different perspective regarding the use of engineering concepts, with students applying concepts to achieve a broad, high-level solution rather than a low-level formulaic, rote approach{Shuman:2008tq}.
2)	Encourage students to work collaboratively and cooperatively as a group, honing teamwork and interpersonal skills and delivering a higher quality solution than individual submissions{Gokhale:1995vd}. 
3)	Encourage integration and synthesis of information and concepts spanning engineering and other disciplines{Yildirim:2010tx}.
4)	Encourage reasoning and higher-order thinking skills through the ill-structured and complex nature of MEA instruction{Chamberlin:2002wk}.

