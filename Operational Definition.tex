\section{2. Operational Definition of Critical Thinking in Engineering}

In order to provide a common point of reference, and to offer a means to determine how well aligned an assessment method is with how critical thinking is applied within engineering, we present the following framework presented below in Figure 2.

It draws from the substantial literature, drawing broadly from international agreements of professional engineering profiles\cite{internationalengineeringalliance}, initiatives in engineering education\cite{crawley2011cdio}, ways of thinking in engineering (design and systems thinking) \cite{Dorst:2010tc, Dym:2005gy, Frank:2001ut}and ways of reasoning in engineering \cite{Stein:2011hr, Paul:2006kv}. 

Underlying the majority of models of engineering thinking is the application of thinking and reasoning to solve complex engineering problems. These types of problems are representative of design problems routinely encountered in professional practise, and can be characterized by the following: 

{IEA Defintion of complex problems}

Consistent with the above, our framework views thinking and reasoning in engineering is conducted within a problem solving context and should embody the three distinguishing characteristics of engineering learning: the use of representations, alignment with professional practices, and emphasis on design\cite{Johri:2014tk}. This context, along with alignment with disciplinary perspectives on learning should naturally lead to the assessment of critical thinking in engineering in an authentic fashion; which emphasizes design principles to develop solutions to  complex engineering problems, which are well aligned and representative of professional practise.

