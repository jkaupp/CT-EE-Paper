\subsubsection{The International Critical Thinking Test}

The International Critical Thinking Test (ICTT) is an essay-style test designed to provide an assessment of the fundamentals of critical thinking. The ICTT has two areas of focus. The first is to provide a reasonable way to measure CTS, while the second is to provide a test instrument that stimulates the faculty to teach their discipline in a manner that fosters critical thinking in the students/cite{Paul:2010vr}. The ICTT is divided into two separate forms: an analysis of a writing prompt and an assessment of the writing prompt. In the analysis segment (Form A) of the test, the student must accurately identify the elements of reasoning within a prompt. In the assessment segment of the test (Form B), the student must critically analyze and evaluate the reasoning used in the original prompt. Student responses are graded according to a rubric based on the elements of reasoning that comprise Paul’s model of critical thinking \cite{Paul:2006kv}:

\begin{enumerate}
\item Purpose
\item Questions
\item Information
\item Conclusions
\item Concepts
\item Assumptions
\item Implication
\item Point of view
\end{enumerate}

The ICTT was authored to have high consequential validity, such that the consequence of using the test would be significant and highly visible to instructors\cite{Paul:2007um}. This encourages discipline-specific adoption of critical thinking and the redevelopment of curriculum that “teach to the test.”

The Paul-Elder model, developed originally by Richard Paul and further refined by both Paul and Elder/cite{Paul:2006kv}. The Paul-Elder model is based on the following working definition of critical thinking as:

that mode of thinking — about any subject, content, or problem — in which the thinker improves the quality of his or her thinking by skillfully analyzing, assessing, and reconstructing it. Critical thinking is self-directed, self-disciplined, self-monitored, and self-corrective thinking. It presupposes assent to rigorous standards of excellence and mindful command of their use. It entails effective communication and problem-solving abilities, as well as a commitment to overcome our native egocentrism and sociocentrism/cite{Paul:2006kv}.

The Paul-Elder model divides critical thinking into three key components: elements of reasoning, intellectual standards and intellectual traits. The elements of reasoning are universal elements that inform and describe all reasoning or thought. The intellectual standards are standards applied to elements of reasoning or thought to interpret or assess quality. Lastly, the intellectual traits are desired traits or characteristics of a skilled practitioner of critical thinking. These three components are interrelated and each contributes to the development of a critical thinker. In the Paul-Elder model, critical thinkers apply the intellectual standards to the elements of reasoning in order to develop intellectual traits (Figure 5). There are two essential dimensions of thinking that students need to master in order to learn how to upgrade their thinking. They need to be able to identify the component parts of their thinking, and they need to be able to assess their use of these parts of thinking. These two essential dimensions, in concert with the intellectual standards, elements of thought and intellectual traits, can be organized into a rubric for the evaluation of critical thinking.
