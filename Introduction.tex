\section{Introduction}


Engineers solve problems.  From simple to complex, constrained to open-ended, everyday to specific, solving problems is one of the fundamental abilities of the discipline.  Underlying problem solving is the reflective, self-regulated judgement, analysis and reasoning exemplified by critical thinking (CT).  Critical thinking is considered by many as an essential outcome of a higher education and is highly valued by employers, educators and policymakers.  Yet, many  programs claim to develop critical thinking skills, but this development is informal, non-continuous, holistic in nature and infrequently assessed\cite{Paul:1997ti,Ahern:2012ey}.  This is usually attributed to the lack of consensus and debates amongst CT scholars and researchers on a number of issues.  These range from a lack of a consensus operational definition, disagreement on domain-specificity\cite{Ennis:1989bm,McPeck:1990is}, effective instructional techniques\cite{Tsui:2002cc} and the difficulty of assessing a latent, complex cognitive and metacognitive skill\cite{Ennis:1993us,Halpern:2003tt}
 
One of the foremost of these arguments is domain-specificity, with two distinct sides.  One side of the argument views critical thinking as a general skill, championed by Ennis\cite{Ennis:1989bm}.  The other views of critical thinking explicitly tied to a specific domain through content and context, championed by McPeck\cite{McPeck:1990is}.  Underpinning the argument are two interrelated aspects : mastery and transfer, or the ability for students to apply skills to novel contexts. The generalists view critical thinking as general in nature, transferable to a variety of contexts upon mastery.  The specifists hold that mastery of a skill does not reliably imply transfer to novel contexts, and critical thinking should be taught in the context of a domain to facilitate both mastery and transfer.  

This conflict further complicates the teaching of critical thinking skills in higher education, and especially in professional programs like engineering.  Critical thinking instruction through a separate general education course is impractical in an accredited discipline bound by curricular requirements.  While these general courses may be effective in developing mastery of CT, they presume students’ ability to transfer knowledge to novel or specific contexts, which is rarely spontaneous occurrence\cite{Abrami:2008td}.

Research into the science of teaching and learning shows that mastery and transfer is best achieved by making the distinct components of a skill explicit, employing instructional strategies that build on an understanding of deep structures and underlying principles of a discipline and how those skills apply and provide diverse contexts to help students learn to transfer skills. This can be accomplished by incorporating CT into individual disciplines through discipline-related frameworks, defined as the distinctive conceptual structures and methodological norms that guide inquiry and shape theory in a given discipline, and providing instruction that combines concrete experience within a discipline with abstract knowledge that crosscuts context\cite{Ambrose:2010uh}. Within engineering the abstract knowledge is the explicit instruction general critical thinking skills according to a framework, and the concrete experience is the understanding of deep structures and principles of the discipline\cite{Smith:2002df}.  

The above approach is consistent with the ‘infusion’ and ‘mixed’ modes of CT instruction, as per Ennis’ typology of critical thinking instruction, described in figure X.  Several meta-analyses of CT research illustrated that explicit instruction is of critical importance, with infusion and mixed producing greater effects than the general or immersion modes\cite{Abrami:2008td ,Abrami:2014cs,BeharHorenstein:2011tf}.  