\subsubsection{Critique of the Cornell-Illinois Model & The Cornell Critical Thinking Test: Level Z}

The Cornell-Illinois framework, presents a vague picture of critical thinking as a set of cognitive skills that are applied to form a course of action.  However, the type of thinking required in solving complex engineering problems is not linear in nature, requiring continual assessment, reflection and monitoring.  These concerns have been addressed by Ennis in subsequent work/cite{Ennis:1993us}, but raise important concerns about the alignment and suitability of the Cornell-Illinois model for use within engineering.  

There are some potential issues with using a multiple-choice assessment of CTS, arising from the fact that the test does not assess dispositional aspects of critical thinking, or how individuals chose to engage in critical thinking.  Multiple choice CT have been criticized as tests assessing verbal and quantitative knowledge and not critical thinking, since the format prevents test-takers from applying CTS to develop their own solution to the problem/cite{Abrami:2008ew,Ku:2009fg}. Additionally, multiple choice tests can only narrowly assess a single concept of thought in a question/cite{Bensley:2012gx,Ku:2009fg}.  This is opposed to the real-world application of critical thinking to solve complex engineering problems which an individual employs a wide variety of concepts and skills to provide a comprehensive solution to a complex, interconnected problem encountered in engineering.  There is also a significant misalignment between the tasks presented in the CCTT and tasks related in solving a complex engineering problem, engineering problems will seldom be as simple as selecting the appropriate response out of a list of possibilities.  While that may exist at some point in solving engineering problems, it is the result of careful and well-reasoned analysis.