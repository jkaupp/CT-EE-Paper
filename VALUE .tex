\subsection{VALUE Rubrics}

The VALUE Rubrics were initially developed through a project launched by the American Association of Colleges and Universities (AAC&U)  to provide a valid assessment of learning in undergraduate education.  These rubrics are broad, discipline-neutral descriptions of selected essential learning outcomes of undergraduate education.  In each rubric, common themes were identified for a particular and performance criteria were developed by panels of experts.  The efforts of the experts were focused on:

Performance criteria focuses on positive demonstration of outcomes rather than what was lacking
Performance criteria can be used to assess to non-traditional modes of artifacts demonstrating student learning
Performance criteria are developed to assess summative displays of student learning rather than developmental or formative displays
Performance criteria are phrased in a manner as to be easily understood by non-experts

There are four levels of performance criteria, from the benchmark level of a student entering university to the capstone level of a student who has just completed their undergraduate experience.   While the performance criteria and levels represent a consensus of experts and can be used in their original form, the rubrics are meant to be modified to foster alignment between course, program or institutional outcomes and to reflect the specific context in which they are used.  