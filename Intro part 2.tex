But what good is a highly effective instructional strategy, without the means to accurately assess it?

There are many assessment approaches and tools for assessing critical thinking in engineering, each developed according to a different framework and definition of critical thinking.  The majority these assessments are standardized and possess excellent psychometric measures of reliability and validity.  However, the generalized nature of these instruments may not reflect how critical thinking skills are applied in engineering.  This misalignment raising questions about the usefulness and accuracy of the results.  Instructors take care to consider additional factors beyond psychometrics, as there are many advantages and disadvantages to popular critical thinking assessments\cite{Ku_2009}

These varied factors can make it incredibly daunting for an instructor to select an assessment approach from a wide array of reliable and valid tools.  This contributes to instructors adopting an ‘easy’ approach that assesses CT, rather than the ‘best-suited’ approach.  If the ultimate goal is the development of critical thinking skills, then an assessment tool must be authentic, practical, sustainable and aligned with curricular and disciplinary goals.  This highlights the importance of assessing skills, be they general or specific, in an authentic fashion within a disciplinary context.

Therefore, in order to help engineering instructors with the difficult task of selecting an assessment approach that best suits their needs we have set out to investigate CT assessment methods in engineering.  Popular CT assessments are presented below, alongside the framework or operational definition of critical thinking upon which the assessment is based.  Also included is an approach to assessing critical thinking, developed by the authors, based on model eliciting activities\cite{DiefesDux:2004fq, Kaupp:2014un}. 

