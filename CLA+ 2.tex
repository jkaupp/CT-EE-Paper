The CLA model relies on a criterion sampling approach that is relatively straightforward and seeks to determine the abilities of a student by sampling tasks from the domain in which the student is to be measured, observing their response and inferring performance and learning on the larger construct. Shavelson (2008) explains criterion sampling by using the example of driving a car:

\begin{quote}
"For example, if you want to know whether a person not only knows the laws that govern driving a car but also if she can actually drive a car, don’t just give her a multiple-choice test. Rather, also administer a driving test with a sample of tasks from the general driving domain such as starting the car, pulling into traffic, turning right and left in traffic, backing up, and parking. Based on this sample of performance, it is possible to draw valid inferences about her driving performance more generally."
\end{quote}

The CLA follows the criterion sampling approach by presenting students with holistic, real-world problems. Through these problems, it samples tasks and collects students’ responses, which are then graded according to a set of generic skills and formed into rubrics. In order to generate a successful response to the task, students would have to apply problem solving successfully, reason analytically, and write convincingly and effectively. Since these are all underlying components of critical thinking as defined by the CLA model, critical thinking ability can thus be inferred from student responses to test questions.

The Collegiate Learning Assessment (CLA) was developed and administered by the Council for Aid to Education (CAE).  The CLA is constructed using the CLA model of critical thinking and problem solving as a foundation. Student response are graded using a series of grading rubrics, and are scored automated system on the on the following scales\cite{Shavelson:2008vo}:

\begin{enumerate}
\item Analytic reasoning
\end{enumerate}
\begin{enumerate}
\item Problem solving
\end{enumerate}
\begin{enumerate}
\item Writing mechanics
\end{enumerate}
\begin{enumerate}
\item Writing effectiveness
\end{enumerate}

\subsection{Critique of The Collegiate Learning Assessment Plus}


The CLA model is not an explicit framework, unlike the Paul-Elder or Cornell-Illinois models, which reduces critical thinking into constituent parts. Rather, the CLA views critical thinking in the broadest sense, as summarized by{Bok:2006wt}:

\begin{quote}
"The ability to think critically—ask pertinent questions, recognize and define problems, identify arguments on all sides of an issue, search for and use relevant data and arrive in the end at carefully reasoned judgments—is the indispensable means of making effective use of information and knowledge."
\end{quote}

This is consistent with the definition of critical thinking as applied to solve complex engineering problems, but lacks a defined structure to be used as an instructional strategy for critical thinking development. 

The CLA consists of two distinct tasks, of which students generally complete one: a “performance task” and an “analytic writing task” containing two subtasks, “make an argument” and “critique an argument.” There has also been some concern raised about the holistic assessment methods of the test not accurately measuring the component cognitive skills of critical thinking, and some critique on the grading method of the CLA{Possin:2013ui}. There has also been some concern with the CLA results not being suitable for comparison at the individual student level, with testing results suitable only for institutional level measures{Klein:2009tn}.  A final concern is that the CLA is typically used to assess longitudinal development CT and is not recommended for measurement across a course experience, which affects the sustainable use of the instrument.  Despite these potential challenges, the CLA is a comprehensive assessment, with the tasks requiring the identification, integration and use of multiple skills and critical thinking concepts.  

The CLA is well aligned with the application of critical thinking skills to solve complex engineering problems.  The tasks presented within the CLA are similar in nature to the complex engineering problems.  Given a scenario, supporting information of varying pedigree on which to base analysis, provide a well-reasoned solution, conclusion or recommendation.   While these tasks may not be fully representative of the scale and complexity of engineering problems, they require the application of the same skills involved in the more comprehensive engineering problems.
