The CLA model relies on a criterion sampling approach that is relatively straightforward and seeks to determine the abilities of a student by sampling tasks from the domain in which the student is to be measured, observing their response and inferring performance and learning on the larger construct. Shavelson (2008) explains criterion sampling by using the example of driving a car:

\begin{quote}
"For example, if you want to know whether a person not only knows the laws that govern driving a car but also if she can actually drive a car, don’t just give her a multiple-choice test. Rather, also administer a driving test with a sample of tasks from the general driving domain such as starting the car, pulling into traffic, turning right and left in traffic, backing up, and parking. Based on this sample of performance, it is possible to draw valid inferences about her driving performance more generally."
\end{quote}

The CLA follows the criterion sampling approach by presenting students with holistic, real-world problems. Through these problems, it samples tasks and collects students’ responses, which are then graded according to a set of generic skills and formed into rubrics. In order to generate a successful response to the task, students would have to apply problem solving successfully, reason analytically, and write convincingly and effectively. Since these are all underlying components of critical thinking as defined by the CLA model, critical thinking ability can thus be inferred from student responses to test questions.

The Collegiate Learning Assessment (CLA) was developed and administered by the Council for Aid to Education (CAE).  The CLA is constructed using the CLA model of critical thinking and problem solving as a foundation. Student response are graded using a series of grading rubrics, and are scored automated system on the on the following scales\cite{Shavelson:2008vo}:

\begin{enumerate}
\item Analytic reasoning
\item Problem solving
\item Writing mechanics
\item Writing effectiveness
\end{enumerate}