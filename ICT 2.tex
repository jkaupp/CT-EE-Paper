\subsection{Critique of the ICT}


The Paul-Elder framework presents a discipline neutral view of critical thinking, and provides a comprehensive cognitive and meta-cognitive view of critical thinking through the standards, elements and traits\cite{Paul:2006kv}.  This model is well aligned, suitable framework for use in engineering and has been adapted specifically for engineering \cite{Niewoehner:2008wo} has been used to form a rubric for the evaluation of critical thinking in engineering \cite{Ralston:2011kh,Ralston:2014vc,Ralston:2010vh}, and has been used as a framework within MEA instruction\cite{Kaupp:2013va}. 

There are a few potential challenges that may be encountered with this style of test. First, the prompts task students with the recall-based identification and evaluation of the elements of thought. While these skills are of vital importance within critical thinking, the specific prompts cannot evaluate how students apply CTS in a real-world setting\cite{Bensley:2011gx}.  This highlights a misalignment between the task presented in the ICTT and what is expected in solving complex engineering problems.  While the application of CTS to solve complex engineering problems requires correctly identifying the specific elements involved in critical thinking, the task on the ICTT does not require students to apply these skills in concert to generate a solution or develop a conclusion.  Ultimately, the specificity of the questions may limit the breadth of response in test-takers, leading to a reduced inclination to engage in critical thinking\cite{Taube:1997vu}. Lastly, as with any essay-style or rubric evaluated test, inter-rater reliability (IRR) is a potential issue that should be considered when administering the test on a large scale\cite{Shavelson:1993uu}. 
