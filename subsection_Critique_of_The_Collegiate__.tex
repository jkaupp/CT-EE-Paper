\subsection{Critique of The Collegiate Learning Assessment Plus}

The CLA model is not an explicit framework, unlike the Paul-Elder or Cornell-Illinois models, which reduces critical thinking into constituent parts. Rather, the CLA views critical thinking in the broadest sense, as summarized by{Bok:2006wt}:

\begin{quote}
"The ability to think critically—ask pertinent questions, recognize and define problems, identify arguments on all sides of an issue, search for and use relevant data and arrive in the end at carefully reasoned judgments—is the indispensable means of making effective use of information and knowledge."
\end{quote}

This is consistent with the definition of critical thinking as applied to solve complex engineering problems, but lacks a defined structure to be used as an instructional strategy for critical thinking development. 

The CLA consists of two distinct tasks, of which students generally complete one: a “performance task” and an “analytic writing task” containing two subtasks, “make an argument” and “critique an argument.” There has also been some concern raised about the holistic assessment methods of the test not accurately measuring the component cognitive skills of critical thinking, and some critique on the grading method of the CLA{Possin:2013ui}. There has also been some concern with the CLA results not being suitable for comparison at the individual student level, with testing results suitable only for institutional level measures{Klein:2009tn}.  A final concern is that the CLA is typically used to assess longitudinal development CT and is not recommended for measurement across a course experience, which affects the sustainable use of the instrument.  Despite these potential challenges, the CLA is a comprehensive assessment, with the tasks requiring the identification, integration and use of multiple skills and critical thinking concepts.  

The CLA is well aligned with the application of critical thinking skills to solve complex engineering problems.  The tasks presented within the CLA are similar in nature to the complex engineering problems.  Given a scenario, supporting information of varying pedigree on which to base analysis, provide a well-reasoned solution, conclusion or recommendation.   While these tasks may not be fully representative of the scale and complexity of engineering problems, they require the application of the same skills involved in the more comprehensive engineering problems.
