\subsection{The Collegiate Learning Assessment Plus (CLA+)}

The description of the CLA model and the Collegiate Learning Assessment are based on the original CLA and subsequent versions used since 2000, and not the newest version the CLA+.  The CLA+ offers improvements over the original CLA, addressing many critiques and concerns of the instrument.  However, there appears to be little difference between the two versions regarding the manner in which they measure critical thinking.

The CLA model was developed for the holistic evaluation of critical thinking through problem solving.  The CLA model holds that critical thinking assessment is best approached holistically, arguing that critical thinking cannot be broken down into component parts and measured. Instead, the CLA views the larger construct of critical thinking as being closely connected to and represented by several criteria or skills that students utilize in their responses on the test, as shown in Figure 6.
